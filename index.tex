\documentclass[12pt,a4paper]{book}
\usepackage{polski}
\usepackage[cp1250]{inputenc}
\usepackage{indentfirst}% wcięcie w pierwszym akapicie
\usepackage{fullpage} %wciecie akapitu na 1 cal

% WMP packages - begin

\usepackage{siunitx}
\usepackage{listings}
\usepackage[utf8]{inputenc}
\usepackage[polish]{babel}
\usepackage{lmodern}
\usepackage[T1]{fontenc}
\usepackage[babel=true]{microtype}
\usepackage[hidelinks]{hyperref}

% WMP packages - end


\linespread{1.3} %odstęp miedzy wierszami
%\usepackage{amsfonts} % czcionki
%\usepackage{amssymb}  % symbole matematyczne
%\usepackage[tableposition=top]{caption}
\usepackage{geometry}
\usepackage{graphicx}
\usepackage{graphics}
\geometry{tmargin=2cm,bmargin=2cm,lmargin=3.5cm,rmargin=2.5cm}
%\usepackage{hyphenat}
\usepackage{url}
%\usepackage{listings}
\usepackage{listings}
\usepackage{color}
\usepackage{multirow}
\sloppy

\usepackage[scaled]{uarial}
%\renewcommand*\familydefault{\sfdefault} %% Only if the base font of the document is to be sans serif
%\usepackage[T1]{fontenc}

\lstset{
	basicstyle=\small\ttfamily,
	keywordstyle=\color{keywordcolor}\bfseries,
	commentstyle=\color{commentcolor}\slshape,
	showstringspaces=false,
	tabsize=2,
	numbers=left,	
	numbersep=1.2em,
	stepnumber=1,
	xleftmargin=1.2em,
	xrightmargin=1.2em,
	breaklines=true,	
	breakatwhitespace=true,	
	frameround=fttt,
	frame=BLtr,
	framextopmargin=1ex,
	framexbottommargin=1ex,
	framexleftmargin=1ex,
	framexrightmargin=1ex,
	backgroundcolor=\color{bgcolor},	
	basewidth=0.42em,
	breakautoindent=false,
	belowcaptionskip=1.5ex,
    language = C
}
\definecolor{bgcolor}{rgb}{0.95,0.95,0.95} 		% kolor tla pod kodem
\definecolor{numbercolor}{rgb}{0.35,0.35,0.35} 	% kolor nr linii kodu
\definecolor{codecolor}{rgb}{0.30,0.20,0.20} 	% kolor kodu
\definecolor{keywordcolor}{rgb}{0.00,0.00,0.55}	% kolor slowa kluczowego
\definecolor{commentcolor}{rgb}{0.40,0.40,0.40}	% kolor komentarza w kodzie

%Marginesy dla druku dwustronnego.
%\twoside

\author{my!}
\title{3 Słowa o git}

\begin{document}
\maketitle

\chapter{Inicjalizacja repo}
Aby sklonować repo u siebie na dysku należy wykonać polecenie:
\begin{lstlisting}[language=bash]
  $ git clone ADRES REPO
\end{lstlisting}
np:
\begin{lstlisting}[language=bash]
  $ git clone git@github.com:WMP/GitLab2.git
\end{lstlisting}

Commit zmian:
\begin{lstlisting}[language=bash]
  $ git commit -m "comment"
\end{lstlisting}

Po zainicjalizowaniu repo nalezy dodać upstream do głównego repo, aby móc pobierac zmiany z repo, które forkowaliśmy:
\begin{lstlisting}[language=bash]
  $ git remote add upstream https://github.com/makas9393/lab.git
\end{lstlisting}

Aby odświeżyć zmiany należy wykonać:
\begin{lstlisting}[language=bash]
  $ git fetch
  $ git rebase origin/master
\end{lstlisting}

Rozwiązanie konfliktu:
\begin{lstlisting}[language=bash]
$ git add -A
$ git rebase --continue
$ git merge --continue
$ git commit -m "error"
\end{lstlisting}

\end{document}
